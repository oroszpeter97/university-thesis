\chapter{Introduction}

In recent years, the video game industry has experienced rapid growth, with games becoming increasingly complex and available on a wide range of platforms. This\\expansion has created a demand for software solutions that are not only visually appealing and engaging, but also maintainable, extensible, and reusable. Developing games that run seamlessly on multiple operating systems, such as Windows and Linux, presents unique challenges for software engineers. These challenges include handling platform-specific differences, managing graphical rendering, and ensuring consistent user experiences across devices.

The motivation behind this thesis is to address these challenges by designing and implementing a cross-platform game using C++. C++ remains one of the most popular programming languages in game development due to its performance, flexibility, and extensive ecosystem. By leveraging modern software development principles and design patterns, this project aims to create a game that is not only functional but also serves as a demonstration of best practices in code organization and architecture. The graphical rendering of the game is implemented using the OpenGL library, which is widely used for high-performance graphics applications and supports multiple platforms.

While the gameplay itself is kept relatively simple, the primary focus of the project is on the quality of the codebase. Special attention is given to reusability, extensibility, and maintainability, which are essential attributes for any long-term software project. The thesis also explores the use of libraries and frameworks that facilitate cross-platform development, such as GLFW, which simplifies window management and input handling across different operating systems.

This work is relevant not only to game developers, but also to anyone interested in software engineering and the application of design patterns in real-world projects. By documenting the development process and the decisions made along the way, the thesis provides insights into the practical aspects of building cross-platform applications and highlights common pitfalls and solutions.

Through this thesis, the reader will gain a comprehensive understanding of the challenges and solutions associated with cross-platform game development in C++, as well as practical guidance for applying modern software engineering principles in similar projects.